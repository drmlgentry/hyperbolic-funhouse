\documentclass[12pt,a4paper]{article}
\usepackage[utf8]{inputenc}
\usepackage{amsmath,amssymb,amsfonts}
\usepackage{graphicx}
\usepackage{hyperref}
\usepackage{geometry}
\usepackage{booktabs}
\usepackage{multirow}
\usepackage{appendix}
\usepackage{cite}
\usepackage{amsthm}
\usepackage{bookmark}
\usepackage{tikz}
\usepackage{physics}
\usepackage{siunitx}
\usepackage{algorithm}
\usepackage{algpseudocode}

\geometry{margin=1in}

\newtheorem{theorem}{Theorem}[section]
\newtheorem{lemma}[theorem]{Lemma}
\newtheorem{corollary}[theorem]{Corollary}
\newtheorem{definition}[theorem]{Definition}
\newtheorem{remark}[theorem]{Remark}

\title{The Geometric Origin of Flavor: \\ Golden Ratio, Hyperbolic Holonomy, and CP Violation}
\author{Marvin Gentry \\ 
        Independent Researcher \\
        ORCID: 0009-0006-4550-2663 \\
        \texttt{drmlgentry@protonmail.com}}
\date{\today}

\begin{document}

\maketitle

\begin{abstract}
We present a complete framework for fermion masses and mixing based on $A_5$ modular symmetry with modulus $\tau$ stabilized at the fixed point $\tau_0 = e^{2\pi i/5}$. The golden ratio $\phi = (1+\sqrt{5})/2$ emerges naturally from the residual $\mathbb{Z}_5$ symmetry. Our key innovation: identifying the CP-violating phase $\delta_{CP}$ with \textbf{hyperbolic holonomy}—parallel transport around non-contractible cycles in $\mathbb{H}/\Gamma(5)$ generates $\delta_{CP} = \ang{68.7}$ ($99.9\%$ agreement with $\delta_{CP}^{\text{exp}} = \ang{68.8}$). We provide: (1) rigorous holonomy calculation with Wilson loop integral, (2) string theory embedding in Type IIB with explicit modulus stabilization, (3) corrected neutrino mass ratio $\Delta m_{21}^2/\Delta m_{31}^2 \approx 0.029$ matching data, (4) explicit perturbation theory generating $\theta_{13} = \ang{8.6}$, (5) anomaly cancellation via Green-Schwarz mechanism, and (6) complete numerical verification code. All predictions are testable in current and future experiments.
\end{abstract}

\tableofcontents

\section{Introduction}
\label{sec:intro}
[Previous introduction with emphasis on geometric paradigm]

\section{String Theory Embedding and Modulus Stabilization}
\label{sec:string}
\input{sections/string_embedding.tex}

\section{Geometry of $\mathbb{H}/\Gamma(5)$ and Modular Forms}
\label{sec:geometry}
\input{sections/geometry.tex}

\section{The Universal Golden Matrix $M_0$}
\label{sec:golden_matrix}
\input{sections/golden_matrix.tex}

\section{Hyperbolic Holonomy: Rigorous Calculation}
\label{sec:holonomy}
\input{sections/holonomy_rigorous.tex}

\section{Fermion Hierarchies from Modular Weights}
\label{sec:hierarchies}
\input{sections/hierarchies.tex}

\section{Perturbation Theory for $\theta_{13} \neq 0$}
\label{sec:perturbations}
\input{sections/perturbations.tex}

\section{Corrected Phenomenological Predictions}
\label{sec:pheno}
\input{sections/predictions.tex}

\section{Theoretical Consistency}
\label{sec:consistency}
\input{sections/consistency.tex}

\section{Conclusion and Outlook}
\label{sec:conclusion}
[Previous conclusion]

\appendix
\section{Complete Holonomy Derivation}
\label{app:holonomy_full}
\input{appendices/holonomy_full.tex}

\section{Anomaly Calculation Details}
\label{app:anomaly}
\input{appendices/anomaly.tex}

\section{Numerical Methods and Code Verification}
\label{app:numerical}
\input{appendices/numerical.tex}

\section*{Code Availability}
All code for numerical verification, model exploration, and figure generation is available at:
\[
\href{https://github.com/drmlgentry/hyperbolic-funhouse}{https://github.com/drmlgentry/hyperbolic-funhouse}
\]
Specific functions:
\begin{itemize}
    \item \texttt{src/holonomy.py}: Wilson loop calculation
    \item \texttt{src/diagonalization.py}: $\delta_{CP}$ extraction
    \item \texttt{notebooks/verification.ipynb}: Complete numerical verification
    \item \texttt{tests/test\_predictions.py}: Unit tests against experimental data
\end{itemize}

\begin{thebibliography}{99}
\bibitem{Feruglio:2019} F. Feruglio, Eur. Phys. J. C \textbf{79} (2019) 125.
\bibitem{Kachru:2003} S. Kachru et al., Phys. Rev. D \textbf{68} (2003) 046005.
\bibitem{ParticleDataGroup:2024} Particle Data Group, PTEP \textbf{2024} (2024) 083C01.
\bibitem{NuFIT:2024} NuFIT 5.3 (2024), \url{http://www.nu-fit.org}.
\bibitem{Denef:2005} F. Denef et al., Adv. Theor. Math. Phys. \textbf{9} (2005) 861.
\end{thebibliography}

\end{document}
